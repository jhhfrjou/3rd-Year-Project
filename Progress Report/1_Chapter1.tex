%% ----------------------------------------------------------------
%% Introduction.tex
%% ---------------------------------------------------------------- 
\chapter{Project Goals} 
Conflict Modelling is the use of modelling techniques to simulate battles and calculate their outcomes. The models used to predict the outcomes of battles are generally in two categories, deterministic models and probabilistic models.\cite{Clausen} Deterministic models are easier to understand and predict the outcomes since they will return the same outcome each time, however they have been criticised for being unrealistic because of the effects of uncontrollable factors and other small qualitative factors. That disadvantage of deterministic models is why probabilistic models are used. The use of probabilities results in a unique outcome being found every time the simulations are used. This makes it them harder to predict and analyse, resulting analysis to be done on the average outcome which takes longer.

Using mathematical models to predict the outcomes of specific engagements generally yields inaccurate results since there are number of qualitative factors that can affect the outcome of an engagement that are very different to any model. However, in larger scales and over several engagements these models become more accurate to the true outcomes and therefore can be used by decision makers to reduce the complexity of the vast amounts of information given to them, steering them to an optimal solution faster than manually processing and understanding the information. 

The Lanchester dynamics is one such model.\cite{lanchester1995aircraft} Originally devised in WW1 to model the attrition rates of troop and aerial combat, it has been extended and modified to apply in more complex situations. For example, the ways that asymmetric warfare commonly found in insurgency situations such as the Vietnam war \cite{Schaffer1969} \cite{Schaffer1967} and the war in Afghanistan \cite{Feichtinger2012} and the multi faction combat that is found in civil wars today such as in Syria. \cite{Kress2018} With the increased complexity of the battlefield with many sides having multiple arms on each size the Lanchester dynamics has been adapted to suit these cases. 

Although the multi-arm combat has been hypothesised, \cite{Colegrave1993} the implementations are static and the result in “wasted firepower” since the constants are not modified when an arm is eliminated. Therefore, modifying the constants in a way to optimise firepower would result in shorter battles and more survivors at the end of the battle. Optimisation of engagements between a single homogenous force attacks a combine arms enemy, \cite{Lin2014} however no research has been published on the fully combined arms combat.

I intend on numerically optimising fully combined arms, heterogenous combat for a single side and simplifying the finding into a useable model that can be used to aid decision making in the future.
